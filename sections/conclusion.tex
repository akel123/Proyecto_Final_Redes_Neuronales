\section{Conclusiones}
En este trabajo se emplearon y analizaron 3 redes neuronales: LeNet, ResNet y una específica para las imágenes de CIFAR-10, donde LeNet fue ligeramente modificada al agregar capas extra para estudiar la overparameterization, y se emplearon los datos de MNIST y CIFAR-10, en el contexto de las imágenes adversarias, con lo que probamos 3 ataques: FGSM, FGM y CW, y una defensa: la compresión JPEG. Vimos que esta defensa efectivamente aumentó la resistencia de la red a los ataques adversarios, pero no por mucho. También existe un compromiso entre la precisión de la red y la calidad de compresión, pues la defensa JPEG tuvo mas éxito con los datos de CIFAR-10 que con los de MNIST, pero también tiene la desventaja de bajar la precisión de la red con rapidez al disminuir la calidad de la imagen con la compresión JPEG, es decir, sin el ruido del ataque adversario.

Las redes que se entrenaron con MNIST tienen una función de transferencia bastante plana, lo cual significa que son igual de sensibles a las frecuencias altas y a las bajas. También vimos que el ruido adversario tiene más frecuencias altas presentes. Entonces una hipótesis interesante es que una red que sea menos sensible a las frecuencias altas será más resistente a los ataques adversarios. No obstante, con las imágenes de CIFAR-10, vimos que el ruido es tan sutil que ni siquiera aparece en la FFT de la imagen adversaria. Parece que entre más compleja sea la imagen (más pixeles), más sutil puede ser el ataque. Al agregar compresión JPEG como una primera capa antes de obtener las funciones de transferencia, no hubo una gran diferencia en la curva resultante. Eso podría haber sido un artefacto del pequeño tamaño de los datos de MNIST. Falta repetir el análisis de las funciones de transferencia con la red de CIFAR-10.

Contrario a lo reportado en la literatura, en el contexto de imágenes médicas \cite{ma2020understanding}, observamos que nuestras redes con overfitting y overparameterization tienen mayor resistencia a los ataques adversarios. Esto podría deberse a la complejidad de las imágenes médicas, pero habría que continuar con el análisis en esa dirección.

El análisis de saliency con los datos de MNIST demostró que con datos así de sencillos, en general, las partes que son más atacadas por el ruido adversario son las partes que al ser modificadas transforman a un número por otro, por ejemplo, al quitar la parte de arriba a la izquierda de un 9 se obtiene un 3. También se nota una diferencia entre el saliency de la red con topología lineal (LeNet) y la red con topología no lineal (ResNet). Hay más regiones con intensidad alta en el gradiente de ResNet que en el de LeNet. Esto podría ayudar a explicar por qué la red con topología no lineal es más susceptible a los ataques.

El análisis de saliency en los datos de CIFAR-10 no es tan intuitivo, probablemente debido a que las imágenes son un poco más complejas que las de MNIST, pero sí se observa una variación en la distribución de saliency conforme aumenta el ruido adversario.