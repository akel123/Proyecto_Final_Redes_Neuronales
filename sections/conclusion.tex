\section{Conclusión/Discusión}
En este reporte, se estudian varias redes y diferentes tipos de datos en el contexto de las imagenes adversarias. También la compresión JPEG fue evaluada como una defensa. Vimos que esta defensa sí mejoró la resistencia de la red a los ataques adversarios, pero no por mucho. También existe un compromiso entre la precisión de la red y la calidad de compresión, aún más para los datos CIFAR-10. Las redes que se entrenaron con MNIST tienen una función de transferencia bastante plana lo cual significa que son igual de sensibles a las frecuencias altas y a las bajas. También vimos que el ruido adversario tiene más altas frecuencias presentes. Entonces una hipótesis interesante sería que hacer que una red fuera menos sensibles a las frecuencias altas haria que tal red ser mas resistente a los ataques adversarios. No obstante, con las imagenes de CIFAR-10, vimos que el ruido ni aparece en la FFT de la imagen adversaria por ser tan sútil. Parece que entre más compleja sea la imagen (más píxeles), más sútil puede ser el ataque. Con respecto a las funciones de transferencia, agregando la compresión JPEG como una primera capa no pareció hacer una gran diferencia en como salió la curva. Eso podría haber sido un artefacto del pequeño tamaño de los datos MNIST. Todavía no se ha repetido el análisis de las transferencias con la red de CIFAR-10.

La defensa JPEG tuvo mas éxito con los datos de CIFAR-10, pero con la desventaja de bajar la precisión de la red más rápido cuando se bajaba la calidad de compresión JPEG.

El análisis de salience con los datos MNIST demostró que con datos así sencillas, el mayoría del tiempo, las partes que son más atacadas por el ruido adversario son las partes que cambian un número a otro (por ejemplo la parte arriba de un ``9''). Sin embargo, en los datos CIFAR-10...